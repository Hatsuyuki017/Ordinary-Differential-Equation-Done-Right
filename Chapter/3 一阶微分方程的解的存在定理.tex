\chapter{一阶微分方程的解的存在定理}

\section{解的存在唯一性定理与逐步逼近法}

\subsection{存在唯一性定理}

考虑一阶微分方程初值问题:
\[
\begin{cases}
\frac{dy}{dx} = f(x, y) \\
\varphi(x_0) = y_0
\end{cases}
\]
其中区域 \( R: |x - x_0| \leq a,\ |y - y_0| \leq b \)。


\begin{theorem}
若函数 \( f(x, y) \) 满足以下条件:
\begin{enumerate}
    \item \( f(x, y) \in C(R) \)(在区域 \( R \) 上连续)
    \item \( f(x, y) \) 在 \( R \) 内关于 \( y \) 满足 Lipschitz 条件:存在正数 \( L \geq 0 \),对任意 \( (x, y_1), (x, y_2) \in R \),有 \( |f(x, y_1) - f(x, y_2)| \leq L|y_1 - y_2| \)
\end{enumerate}

则方程在区间 \( x \in [x_0 - h_0, x_0 + h_0] \) 上必存在唯一解 \( y = \varphi(x) \),且满足 \( \varphi(x_0) = y_0 \)。

其中 \( h_0 = \min\left(a, \frac{b}{M}\right) \),\( M = \max_{(x,y) \in R} |f(x, y)| \)。
\end{theorem}

\begin{remark}
    \begin{enumerate}
        \item 若 \( f(x, y) \) 关于 \( y \) 的偏导数存在且有界(或 \( f_y(x, y) \) 连续),则 \( f(x, y) \) 满足 Lipschitz 条件。
        \item 几何解释:解曲线 \( y = \varphi(x) \) 过初值点 \( (x_0, y_0) \) 且完全包含在区域 \( R \) 内。
    \end{enumerate}
\end{remark}

\begin{proposition}
求上述初值问题的解,等价于求积分方程 \( y(x) = y_0 + \int_{x_0}^x f(\xi, y(\xi)) d\xi \) 在区间 \( x \in [x_0 - h_0, x_0 + h_0] \) 上的连续解。
\end{proposition}

\begin{proof}
$(\Rightarrow)$ 设 \( y = \varphi(x) \) 是 \( \frac{dy}{dx} = f(x, y) \) 的解,且 \( \varphi(x_0) = y_0 \)。由导数定义,\( \varphi'(x) = f(x, \varphi(x)) \),两边从 \( x_0 \) 到 \( x \) 积分:
\[
\varphi(x) - \varphi(x_0) = \int_{x_0}^x \varphi'(\xi) d\xi = \int_{x_0}^x f(\xi, \varphi(\xi)) d\xi
\]
代入 \( \varphi(x_0) = y_0 \),得 \( \varphi(x) = y_0 + \int_{x_0}^x f(\xi, \varphi(\xi)) d\xi \),且 \( \varphi(x) \in C[x_0 - h_0, x_0 + h_0] \)。

$(\Leftarrow)$ 设 \( \varphi(x) = y_0 + \int_{x_0}^x f(\xi, \varphi(\xi)) d\xi \) 连续,则由变上限积分求导法则:
\[
\varphi'(x) = \left( \int_{x_0}^x f(\xi, \varphi(\xi)) d\xi \right)' = f(x, \varphi(x))
\]
且 \( \varphi(x_0) = y_0 \),故 \( \varphi(x) \) 是初值问题的解。
\end{proof}


\subsubsection{构造逐步逼近序列 \( \{\psi_n(x)\} \)}
通过以下递推公式定义逼近序列:
\[
\begin{cases}
\psi_0(x) = y_0 \\
\psi_n(x) = y_0 + \int_{x_0}^x f(\xi, \psi_{n-1}(\xi)) d\xi, \quad n = 1, 2, \dots
\end{cases}
\]
其中 \( \psi_0(x) = y_0 \) 是过初值点 \( (x_0, y_0) \) 的常值函数。


\begin{proposition}
对任意正整数 \( n \),\( \psi_n(x) \) 在 \( x \in [x_0 - h_0, x_0 + h_0] \) 内有定义且连续,且满足 \( |\psi_n(x) - y_0| \leq b \)(即 \( \psi_n(x) \subset R \))。
\end{proposition}

\begin{proof}
用数学归纳法证明:
\begin{enumerate}
    \item 当 \( n = 0 \) 时,\( \psi_0(x) = y_0 \),显然满足 \( |\psi_0(x) - y_0| = 0 \leq b \),连续且在区间内有定义。
    \item 假设 \( n = k-1 \) 时,\( \psi_{k-1}(x) \) 满足结论(即 \( |\psi_{k-1}(x) - y_0| \leq b \),连续且有定义)。对 \( n = k \):
\[
|\psi_k(x) - y_0| = \left| \int_{x_0}^x f(\xi, \psi_{k-1}(\xi)) d\xi \right| \leq \int_{x_0}^x |f(\xi, \psi_{k-1}(\xi))| d\xi \leq M |x - x_0| \leq M h_0 \leq b
\]
且 \( \psi_k(x) \) 由连续函数的积分定义,故连续且在区间内有定义。
\end{enumerate}
由归纳法,对所有 \( n = 1, 2, \dots \),命题成立。
\end{proof}


\begin{proposition}
序列 \( \{\psi_n(x)\} \) 在 \( x \in [x_0 - h_0, x_0 + h_0] \) 上一致收敛。
\end{proposition}

\begin{proof}
将 \( \psi_n(x) \) 表示为级数形式:
\[
\psi_n(x) = \psi_0(x) + \sum_{k=1}^n [\psi_k(x) - \psi_{k-1}(x)]
\]
只需证明级数 \( \sum_{k=1}^\infty [\psi_k(x) - \psi_{k-1}(x)] \) 一致收敛。

用归纳法证明余项估计:
\begin{enumerate}
    \item 当 \( k = 1 \) 时,\( |\psi_1(x) - \psi_0(x)| \leq \int_{x_0}^x |f(\xi, y_0)| d\xi \leq M |x - x_0| \)
    \item 假设 \( k = m \) 时,\( |\psi_m(x) - \psi_{m-1}(x)| \leq \frac{M L^{m-1} |x - x_0|^m}{m!} \),则对 \( k = m+1 \):
\[
|\psi_{m+1}(x) - \psi_m(x)| \leq \int_{x_0}^x |f(\xi, \psi_m(\xi)) - f(\xi, \psi_{m-1}(\xi))| d\xi \leq L \int_{x_0}^x |\psi_m(\xi) - \psi_{m-1}(\xi)| d\xi \leq \frac{M L^m |x - x_0|^{m+1}}{(m+1)!}
\]
\end{enumerate}
由于 \( |x - x_0| \leq h_0 \),故 \( |\psi_k(x) - \psi_{k-1}(x)| \leq \frac{M L^{k-1} h_0^k}{k!} \)。而级数 \( \sum_{k=1}^\infty \frac{M L^{k-1} h_0^k}{k!} = \frac{M}{L} (e^{L h_0} - 1) \) 收敛,由 Weierstrass 判别法,原级数一致收敛,即 \( \{\psi_n(x)\} \) 一致收敛。
\end{proof}


\begin{proposition}
设 \( \psi(x) = \lim_{n \to \infty} \psi_n(x) \),则 \( \psi(x) \) 是积分方程 \( y(x) = y_0 + \int_{x_0}^x f(\xi, y(\xi)) d\xi \) 的连续解。
\end{proposition}

\begin{proof}
由 \( \{\psi_n(x)\} \) 一致收敛且每一项连续,故 \( \psi(x) \) 连续。对逼近公式两边取极限:
\[
\lim_{n \to \infty} \psi_n(x) = y_0 + \lim_{n \to \infty} \int_{x_0}^x f(\xi, \psi_{n-1}(\xi)) d\xi
\]
由一致收敛性,积分与极限可交换:
\[
\psi(x) = y_0 + \int_{x_0}^x f(\xi, \lim_{n \to \infty} \psi_{n-1}(\xi)) d\xi = y_0 + \int_{x_0}^x f(\xi, \psi(\xi)) d\xi
\]
故 \( \psi(x) \) 是积分方程的连续解。
\end{proof}


\begin{proposition}
设 \( \psi(x) \) 和 \( \varphi(x) \) 都是积分方程在 \( [x_0 - h_0, x_0 + h_0] \) 上的连续解,则 \( \psi(x) = \varphi(x) \)(解的唯一性)。
\end{proposition}

\begin{proof}
由积分方程定义:
\[
\psi(x) - \varphi(x) = \int_{x_0}^x [f(\xi, \psi(\xi)) - f(\xi, \varphi(\xi))] d\xi
\]
取绝对值并利用 Lipschitz 条件:
\[
|\psi(x) - \varphi(x)| \leq L \int_{x_0}^x |\psi(\xi) - \varphi(\xi)| d\xi
\]
令 \( u(x) = \int_{x_0}^x |\psi(\xi) - \varphi(\xi)| d\xi \),则 \( u'(x) = |\psi(x) - \varphi(x)| \leq L u(x) \),且 \( u(x_0) = 0 \)。

由 Gronwall 不等式,\( u(x) \leq 0 \),而 \( u(x) \geq 0 \),故 \( u(x) = 0 \),即 \( \psi(x) = \varphi(x) \)。
\end{proof}


\subsubsection{一阶线性方程的存在唯一性验证}
考虑一阶线性方程 \( \frac{dy}{dx} = P(x)y + Q(x) \),其中 \( P(x), Q(x) \) 在区间 \( I \) 上连续。

令 \( f(x, y) = P(x)y + Q(x) \),则:
\[
|f(x, y_1) - f(x, y_2)| = |P(x)(y_1 - y_2)| \leq L |y_1 - y_2|
\]
其中 \( L = \max_{x \in I} |P(x)| \)(因 \( P(x) \) 连续,故 \( L \) 存在)。因此,该方程满足存在唯一性定理条件,在 \( I \) 上存在唯一解。


\begin{theorem}
考虑一阶隐式微分方程 \( F(x, y, y') = 0 \),若在点 \( (x_0, y_0, y_0') \) 的某邻域内满足:
1. \( F(x, y, y') \) 对所有变元 \( (x, y, y') \) 连续,且存在连续偏导数;
2. \( F(x_0, y_0, y_0') = 0 \);
3. \( \frac{\partial F(x_0, y_0, y_0')}{\partial y'} \neq 0 \);

则方程存在唯一解 \( y = \varphi(x) \),定义在 \( |x - x_0| \leq h \)(\( h \) 为足够小的正数)上,且满足初值条件 \( \varphi(x_0) = y_0 \),\( \varphi'(x_0) = y_0' \)。
\end{theorem}


\subsection{近似计算和误差估计}

对逐步逼近序列 \( \{\psi_n(x)\} \),第 \( n \) 次近似解 \( \psi_n(x) \) 与真正解 \( \varphi(x) \) 在 \( |x - x_0| \leq h_0 \) 内的误差估计式为:
\[
|\psi_n(x) - \varphi(x)| \leq \frac{M L^n h_0^{n+1}}{(n+1)!}
\]


\begin{example}
方程 \( \frac{dy}{dx} = x^2 + y^2 \) 定义在矩形域 \( R: -1 \leq x \leq 1,\ -1 \leq y \leq 1 \) 上,确定经过点 \( (0, 0) \) 的解的存在区间,并求误差不超过 \( 0.05 \) 的近似解。
\end{example}

\begin{solution}
\begin{enumerate}
    \item 确定存在区间\\
    \begin{enumerate}
        \item 计算 \( M = \max_{(x,y) \in R} |f(x, y)| = \max (x^2 + y^2) = 1^2 + 1^2 = 2 \)
        \item 区域参数 \( a = 1 \),\( b = 1 \),故 \( h_0 = \min\left(a, \frac{b}{M}\right) = \min\left(1, \frac{1}{2}\right) = \frac{1}{2} \)
        \item 存在区间为 \( x \in \left[-\frac{1}{2}, \frac{1}{2}\right] \)。
    \end{enumerate}
    \item 误差估计与近似解\\
    \begin{enumerate}
        \item 由 \( f_y = 2y \),得 \( L = \max_{(x,y) \in R} |f_y| = 2 \)
        \item 误差估计式简化为:
     \[
     |\psi_n(x) - \varphi(x)| \leq \frac{M}{L} \cdot \frac{(L h_0)^{n+1}}{(n+1)!} = \frac{2}{2} \cdot \frac{(2 \cdot \frac{1}{2})^{n+1}}{(n+1)!} = \frac{1}{(n+1)!}
     \]
     \item 
     要求误差 \( \leq 0.05 \),即 \( \frac{1}{(n+1)!} \leq 0.05 \):
     - \( n = 2 \) 时,\( \frac{1}{3!} = \frac{1}{6} \approx 0.166 > 0.05 \);
     - \( n = 3 \) 时,\( \frac{1}{4!} = \frac{1}{24} \approx 0.0417 < 0.05 \),故取 \( n = 3 \)。
    \end{enumerate}

    \item 计算近似解\\
    \begin{align*}
        \psi_0(x) &= 0\\
        \psi_1(x) &= 0 + \int_0^x (\xi^2 + 0^2) d\xi \\&= \frac{x^3}{3} \\
        \psi_2(x) &= 0 + \int_0^x \left( \xi^2 + \left( \frac{\xi^3}{3} \right)^2 \right) d\xi \\&= \frac{x^3}{3} + \frac{x^7}{63} \\
        \psi_3(x) &= 0 + \int_0^x \left( \xi^2 + \left( \frac{\xi^3}{3} + \frac{\xi^7}{63} \right)^2 \right) d\xi \\&= \frac{x^3}{3} + \frac{x^7}{63} + \frac{2x^{11}}{2079} + \frac{x^{15}}{59535}
    \end{align*}
\end{enumerate}


因此,\( \psi_3(x) \) 是区间 \( \left[-\frac{1}{2}, \frac{1}{2}\right] \) 上误差不超过 \( 0.05 \) 的近似解。
\end{solution}


\section{解的延拓和解对初值的连续性与可微性}

\subsection{解的延拓}

\begin{theorem}[解的延拓定理]
设 \( f(x, y) \) 在区域 \( G \) 内连续,且在 \( G \) 内关于 \( y \) 满足局部 Lipschitz 条件,则方程 \( \frac{dy}{dx} = f(x, y) \) 通过 \( G \) 内任意一点 \( (x_0, y_0) \) 的解 \( y = \varphi(x) \) 可以延拓,直到点 \( (x, \varphi(x)) \) 任意接近 \( G \) 的边界。

以 \( x \) 增大的一方延拓为例:若解只能延拓到区间 \( [x_0, d) \) 上,则当 \( x \to d^- \) 时,\( (x, \varphi(x)) \) 趋于 \( G \) 的边界。
\end{theorem}


\begin{corollary}
若 \( G \) 是无界区域,在延拓定理条件下,方程通过 \( (x_0, y_0) \) 的解 \( y = \varphi(x) \) 向 \( x \) 增大的一方延拓时,有以下两种情况:
1. 解可以延拓到区间 \( [x_0, +\infty) \);
2. 解只能延拓到区间 \( [x_0, d) \)(\( d \) 为有限数),此时当 \( x \to d^- \) 时,要么 \( |\varphi(x)| \to +\infty \),要么 \( (x, \varphi(x)) \) 趋于 \( G \) 的边界。
\end{corollary}


\begin{example}
试论方程 \( \frac{dy}{dx} = \frac{y^2 - 1}{2} \) 分别通过点 \( (0, 0) \) 和 \( (\ln 2, 3) \) 的解的存在区间。
\end{example}

\begin{solution}
\begin{enumerate}
    \item 求通解\\
    方程为可分离变量方程,分离变量得:
   \[
   \frac{2 dy}{y^2 - 1} = dx
   \]
   积分得:
   \[
   \ln \left| \frac{y - 1}{y + 1} \right| = x + \ln |C| \implies \frac{y - 1}{y + 1} = C e^x \implies y = \frac{1 + C e^x}{1 - C e^x}
   \]

   \item 通过 \( (0, 0) \) 的解\\
   代入初值 \( x = 0, y = 0 \),得 \( 0 = \frac{1 + C}{1 - C} \implies C = -1 \),故解为:
   \[
   y = \frac{1 - e^x}{1 + e^x}
   \]
   该函数对所有 \( x \in \mathbb{R} \) 有定义(分母 \( 1 + e^x > 0 \)),故存在区间为 \( (-\infty, +\infty) \)。

    \item 通过 \( (\ln 2, 3) \) 的解\\
    代入初值 \( x = \ln 2, y = 3 \),得 \( 3 = \frac{1 + C e^{\ln 2}}{1 - C e^{\ln 2}} = \frac{1 + 2C}{1 - 2C} \implies C = \frac{1}{4} \),故解为:
   \[
   y = \frac{1 + \frac{1}{4} e^x}{1 - \frac{1}{4} e^x} = \frac{4 + e^x}{4 - e^x}
   \]
   分母为零时 \( 4 - e^x = 0 \implies x = \ln 4 \)。当 \( x \to (\ln 4)^- \) 时,\( y \to +\infty \);当 \( x \to -\infty \) 时,\( y \to 1 \)(在区域内)。故存在区间为 \( (-\infty, \ln 4) \)。
\end{enumerate}
   
\end{solution}

\subsection{解对初值的连续性与可微性}

\begin{theorem}[解对初值的连续性定理]
若函数 \( f(x, y) \) 在区域 \( G \) 内连续,且关于 \( y \) 满足局部 Lipschitz 条件,则方程 \( \frac{dy}{dx} = f(x, y) \) 的解 \( y = \varphi(x; x_0, y_0) \)(其中 \( (x_0, y_0) \) 为初值点)作为 \( x, x_0, y_0 \) 的函数,在其存在范围内是连续的。
\end{theorem}


\begin{theorem}[解对初值的可微性定理]
若函数 \( f(x, y) \) 在区域 \( G \) 内连续,且 \( f_y(x, y) \) 在 \( G \) 内连续,则方程 \( \frac{dy}{dx} = f(x, y) \) 的解 \( y = \varphi(x; x_0, y_0) \) 作为 \( x, x_0, y_0 \) 的函数,在其存在范围内连续可微,且其偏导数满足:
\begin{enumerate}
    \item 对初值 \( x_0 \) 的偏导数:
   \[
   \frac{\partial \varphi}{\partial x_0} = -f(x, \varphi) \exp\left( \int_{x_0}^x f_y(\xi, \varphi(\xi; x_0, y_0)) d\xi \right)
   \]
   \item 对初值 \( y_0 \) 的偏导数:
   \[
   \frac{\partial \varphi}{\partial y_0} = \exp\left( \int_{x_0}^x f_y(\xi, \varphi(\xi; x_0, y_0)) d\xi \right)
   \]
\end{enumerate}
\end{theorem}


\section{奇解}

\subsection{包络的定义与求法}

设给定参数曲线族 \( \Phi(x, y, C) = 0 \),其中 \( C \) 为参数,\( \Phi(x, y, C) \) 是 \( x, y, C \) 的连续可微函数。

\begin{definition}[包络]
曲线族 \( \Phi(x, y, C) = 0 \) 的包络是一条曲线,它满足:
\begin{enumerate}
    \item 本身不包含在曲线族中
    \item 过该曲线的每一点,有曲线族中的一条曲线与它在该点相切
\end{enumerate}

包络包含在由以下方程组消去参数 \( C \) 得到的曲线(称为 \( C \)-判别曲线)之中:
\[
\begin{cases}
\Phi(x, y, C) = 0 \\
\frac{\partial \Phi}{\partial C}(x, y, C) = 0
\end{cases}
\]
\end{definition}
\begin{remark}
    \( C \)-判别曲线中可能包含非包络曲线(如奇点轨迹),需进一步检验。
\end{remark}

\begin{example}
求直线族 \( x \cos \alpha + y \sin \alpha - p = 0 \)(其中 \( \alpha \) 为参数,\( p > 0 \) 为常数)的包络。
\end{example}

\begin{solution}
\begin{enumerate}
    \item 设 \( \Phi(x, y, \alpha) = x \cos \alpha + y \sin \alpha - p = 0 \),对 \( \alpha \) 求偏导:
   \[
   \frac{\partial \Phi}{\partial \alpha} = -x \sin \alpha + y \cos \alpha = 0
   \]

   \item  联立方程组:
   \[
   \begin{cases}
   x \cos \alpha + y \sin \alpha = p \\
   -x \sin \alpha + y \cos \alpha = 0
   \end{cases}
   \]

   \item 消去参数 \( \alpha \):\\
   将两式分别平方后相加:
   \[
   x^2 (\cos^2 \alpha + \sin^2 \alpha) + y^2 (\sin^2 \alpha + \cos^2 \alpha) = p^2 \implies x^2 + y^2 = p^2
   \]
\end{enumerate}
该圆即为直线族的包络。
\end{solution}


\begin{example}
求曲线族 \( (y - C)^2 - \frac{2}{3}(x - C)^3 = 0 \) 的包络。
\end{example}

\begin{solution}
\begin{enumerate}
    \item 设 \( \Phi(x, y, C) = (y - C)^2 - \frac{2}{3}(x - C)^3 = 0 \),对 \( C \) 求偏导:\\
   \[
   \frac{\partial \Phi}{\partial C} = -2(y - C) + 2(x - C)^2 = 0 \implies y - C = (x - C)^2
   \]
   \item  联立方程组并消去 \( C \):\\
   将 \( y = C + (x - C)^2 \) 代入原曲线族方程:
   \[
   [(x - C)^2]^2 - \frac{2}{3}(x - C)^3 = 0 \implies (x - C)^3 \left( (x - C) - \frac{2}{3} \right) = 0
   \]
   - 若 \( (x - C)^3 = 0 \),则 \( C = x \),代入 \( y = C + (x - C)^2 \) 得 \( y = x \)(检验:代入原方程得 \( 0 - \frac{2}{3} \cdot 0 = 0 \),但该曲线是曲线族中 \( C = x \) 的退化情况,非包络);
   - 若 \( (x - C) - \frac{2}{3} = 0 \),则 \( C = x - \frac{2}{3} \),代入 \( y = C + (x - C)^2 \) 得 \( y = x - \frac{2}{3} + \left( \frac{2}{3} \right)^2 = x - \frac{2}{9} \)。

    \item 检验包络:\\
   曲线 \( y = x - \frac{2}{9} \) 不包含在原曲线族中,且过其上每一点均有原曲线族中的曲线相切,故为包络。
\end{enumerate}
\end{solution}


\subsection{奇解的定义与求法}

\begin{definition}[奇解]
若微分方程的某一个解满足:该解上每一点的唯一性不成立(即每一点至少还有另一个解通过),则该解称为方程的奇解。


\end{definition}

\begin{remark}
     \begin{enumerate}
         \item 一阶微分方程通解的包络(若存在)必为奇解
         \item 一阶微分方程的奇解(若存在)必为其通解的包络
     \end{enumerate}
\end{remark}

\subsubsection{奇解的求法}
\begin{enumerate}
    \item 先求方程的通解,再求通解的包络(即 \( C \)-判别曲线),检验后得到奇解
    \item 对一阶隐式方程 \( F(x, y, y') = 0 \),奇解包含在由以下方程组消去 \( p = y' \) 得到的曲线(称为 \( P \)-判别曲线)之中:
\[
\begin{cases}
F(x, y, p) = 0 \\
\frac{\partial F}{\partial p}(x, y, p) = 0
\end{cases}
\]
需检验 \( P \)-判别曲线是否为方程的解,且是否满足奇解定义。
\end{enumerate}

\begin{example}
求方程 \( \left( \frac{dy}{dx} \right)^2 + y^2 - 1 = 0 \) 的奇解。
\end{example}

\begin{solution}
\begin{enumerate}
    \item 用 \( P \)-判别曲线法:\\
    设 \( F(x, y, p) = p^2 + y^2 - 1 = 0 \),对 \( p \) 求偏导:
   \[
   \frac{\partial F}{\partial p} = 2p = 0 \implies p = 0
   \]
   \item 联立方程组并消去 \( p \):\\
    将 \( p = 0 \) 代入 \( F(x, y, p) = 0 \),得 \( y^2 - 1 = 0 \implies y = \pm 1 \)。
    \item 检验:\\
   - 验证 \( y = \pm 1 \) 是方程的解:\( 0^2 + (\pm 1)^2 - 1 = 0 \),满足方程;
   - 方程的通解为 \( y = \sin(x + C) \)(分离变量求解得),\( y = \pm 1 \) 是通解的包络(通解的正弦曲线始终与 \( y = \pm 1 \) 相切),故 \( y = \pm 1 \) 是奇解。
\end{enumerate}
\end{solution}

\begin{example}
求方程 \( y = 2x \frac{dy}{dx} - \left( \frac{dy}{dx} \right)^2 \) 的奇解。
\end{example}

\begin{solution}
\begin{enumerate}
    \item 用 \( P \)-判别曲线法:\\
    设 \( F(x, y, p) = 2x p - p^2 - y = 0 \),对 \( p \) 求偏导:
   \[
   \frac{\partial F}{\partial p} = 2x - 2p = 0 \implies p = x
   \]
   \item 联立方程组并消去 \( p \):\\
   将 \( p = x \) 代入 \( F(x, y, p) = 0 \),得 \( y = 2x \cdot x - x^2 = x^2 \)。
   \item 检验:\\
   验证 \( y = x^2 \) 是否为方程的解:计算 \( y' = 2x \),代入方程右边得 \( 2x \cdot 2x - (2x)^2 = 4x^2 - 4x^2 = 0 \neq x^2 \),故 \( y = x^2 \) 不是方程的解。因此,该方程没有奇解。
\end{enumerate}
\end{solution}

1.文档中的章节不用手动编号,仅将章节名字放在\section{章节名字}中即可
2.最高级为chapter 顺序往下编排
3.文档中其他需要编号的地方采用enumerate环境
4.将公式中的d换为\mathrm{d}
5.注的内容放在remark环境中