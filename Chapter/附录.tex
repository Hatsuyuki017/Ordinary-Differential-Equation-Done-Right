\chapter{附录}
\section{1 绪论}
\section{2 一阶微分方程}
\subsection{伯努利微分方程}
\begin{definition}[伯努利微分方程]
形如
\begin{equation}
    \frac{dy}{dx} + P(x)y = Q(x)y^n \quad (n \neq 0,1)
    \label{eq:bernoulli_form}
\end{equation}
的一阶常微分方程称为\textbf{伯努利微分方程},其中$P(x)$、$Q(x)$为已知连续函数,$n$为实数常数。
\end{definition}

\begin{remark}
当$n=0$或$n=1$时,方程退化为线性微分方程:
\begin{itemize}
    \item 若$n=0$:$\frac{dy}{dx} + P(x)y = Q(x)$
    \item 若$n=1$:$\frac{dy}{dx} + [P(x)-Q(x)]y = 0$
\end{itemize}
这两种情况均可使用一阶线性微分方程的求解方法。
\end{remark}

\subsubsection{求解方法}

伯努利方程可通过变量代换化为线性微分方程求解,具体步骤如下:

\paragraph{步骤1:变量代换}
令
\begin{equation}
    z = y^{1-n}
    \label{eq:substitution}
\end{equation}
则通过求导可得:
\begin{equation}
    \frac{dz}{dx} = (1-n)y^{-n}\frac{dy}{dx}
    \label{eq:dzdx}
\end{equation}

\paragraph{步骤2:方程变换}
将原方程(\ref{eq:bernoulli_form})两边同乘以$(1-n)y^{-n}$:
\begin{equation}
    (1-n)y^{-n}\frac{dy}{dx} + (1-n)P(x)y^{1-n} = (1-n)Q(x)
\end{equation}
代入变量代换关系(\ref{eq:substitution})和(\ref{eq:dzdx}),得到:
\begin{equation}
    \frac{dz}{dx} + (1-n)P(x)z = (1-n)Q(x)
    \label{eq:linear_eq}
\end{equation}

\paragraph{步骤3:求解线性方程}
方程(\ref{eq:linear_eq})是关于$z$的一阶线性微分方程,其通解为:
\begin{equation}
    z = e^{-\int (1-n)P(x)dx} \left[ \int (1-n)Q(x)e^{\int (1-n)P(x)dx}dx + C \right]
    \label{eq:linear_solution}
\end{equation}
其中$C$为任意常数。

\paragraph{步骤4:回代原变量}
最后将$z = y^{1-n}$代回,得到原方程的通解:
\begin{equation}
    y^{1-n} = e^{-\int (1-n)P(x)dx} \left[ \int (1-n)Q(x)e^{\int (1-n)P(x)dx}dx + C \right]
    \label{eq:final_solution}
\end{equation}

\subsubsection{应用实例}

\begin{example}[人口增长模型]
考虑包含竞争效应的人口增长模型:
\begin{equation}
    \frac{dN}{dt} = rN - kN^2
\end{equation}
其中$r$为固有增长率,$k$为竞争系数。该方程可写为伯努利方程形式:
\begin{equation}
    \frac{dN}{dt} - rN = -kN^2
\end{equation}
这里$n=2$,令$z = N^{1-2} = N^{-1}$,可化为线性方程求解。
\end{example}

\begin{example}[流体力学应用]
在理想流体力学中,伯努利方程描述了沿流线的能量守恒:
\begin{equation}
    \frac{dv}{ds} + \frac{1}{\rho}\frac{dp}{ds} + g\frac{dz}{ds} = 0
\end{equation}
在某些特定条件下可化为微分方程形式。
\end{example}

\subsubsection{理论意义}

伯努利微分方程在常微分方程理论中具有重要地位:
\begin{itemize}
    \item 展示了一类非线性方程线性化的典型方法
    \item 提供了变量代换法的经典范例
    \item 在物理学、生物学、经济学等领域有广泛应用
    \item 是研究更复杂非线性微分方程的基础
\end{itemize}

\subsection{常用积分}
\begin{longtable}{clc}
\caption{24个常用不定积分公式} \\
\toprule
序号 & $\displaystyle \int f(x) \, dx$ & $F(x) + C$ \\
\midrule
\endfirsthead

\multicolumn{3}{c}{\textbf{表格续前页}} \\
\toprule
序号 & $\displaystyle \int f(x) \, dx$ & $F(x) + C$ \\
\midrule
\endhead

\midrule
\multicolumn{3}{r}{\textit{续下页}} \\
\endfoot

\bottomrule
\endlastfoot

1 & $\displaystyle \int k \, dx$ & $kx + C$ \\[8pt]
2 & $\displaystyle \int x^n \, dx$ & $\displaystyle \frac{x^{n+1}}{n+1} + C \quad (n \neq -1)$ \\[8pt]
3 & $\displaystyle \int \frac{1}{x} \, dx$ & $\ln|x| + C$ \\[8pt]
4 & $\displaystyle \int e^x \, dx$ & $e^x + C$ \\[8pt]
5 & $\displaystyle \int a^x \, dx$ & $\displaystyle \frac{a^x}{\ln a} + C \quad (a > 0, a \neq 1)$ \\[8pt]
6 & $\displaystyle \int \sin x \, dx$ & $-\cos x + C$ \\[8pt]
7 & $\displaystyle \int \cos x \, dx$ & $\sin x + C$ \\[8pt]
8 & $\displaystyle \int \tan x \, dx$ & $-\ln|\cos x| + C$ \\[8pt]
9 & $\displaystyle \int \cot x \, dx$ & $\ln|\sin x| + C$ \\[8pt]
10 & $\displaystyle \int \sec x \, dx$ & $\ln|\sec x + \tan x| + C$ \\[8pt]
11 & $\displaystyle \int \csc x \, dx$ & $\ln|\csc x - \cot x| + C$ \\[8pt]
12 & $\displaystyle \int \sec^2 x \, dx$ & $\tan x + C$ \\[8pt]
13 & $\displaystyle \int \csc^2 x \, dx$ & $-\cot x + C$ \\[8pt]
14 & $\displaystyle \int \sec x \tan x \, dx$ & $\sec x + C$ \\[8pt]
15 & $\displaystyle \int \csc x \cot x \, dx$ & $-\csc x + C$ \\[8pt]
16 & $\displaystyle \int \frac{1}{\sqrt{1-x^2}} \, dx$ & $\arcsin x + C$ \\[8pt]
17 & $\displaystyle \int \frac{1}{1+x^2} \, dx$ & $\arctan x + C$ \\[8pt]
18 & $\displaystyle \int \frac{1}{x\sqrt{x^2-1}} \, dx$ & $\text{arcsec}\, x + C$ \\[8pt]
19 & $\displaystyle \int \frac{1}{\sqrt{a^2-x^2}} \, dx$ & $\displaystyle \arcsin \frac{x}{a} + C$ \\[8pt]
20 & $\displaystyle \int \frac{1}{a^2+x^2} \, dx$ & $\displaystyle \frac{1}{a}\arctan \frac{x}{a} + C$ \\[8pt]
21 & $\displaystyle \int \frac{1}{x^2-a^2} \, dx$ & $\displaystyle \frac{1}{2a}\ln\left|\frac{x-a}{x+a}\right| + C$ \\[8pt]
22 & $\displaystyle \int \frac{1}{\sqrt{x^2 \pm a^2}} \, dx$ & $\displaystyle \ln\left|x + \sqrt{x^2 \pm a^2}\right| + C$ \\[8pt]
23 & $\displaystyle \int \sqrt{a^2-x^2} \, dx$ & $\displaystyle \frac{x}{2}\sqrt{a^2-x^2} + \frac{a^2}{2}\arcsin\frac{x}{a} + C$ \\[8pt]
24 & $\displaystyle \int \sqrt{x^2 \pm a^2} \, dx$ & $\displaystyle \frac{x}{2}\sqrt{x^2 \pm a^2} \pm \frac{a^2}{2}\ln\left|x + \sqrt{x^2 \pm a^2}\right| + C$ \\[8pt]

\end{longtable}

\subsection{欧拉公式}
\begin{equation}
    e^{i\theta}=\cos\theta+i\sin\theta 
\end{equation}
\begin{align}
    \cos\theta=\frac{e^{i\theta}+e^{-i\theta}}{2}\\
    \sin\theta=\frac{e^{i\theta}-e^{-i\theta}}{2}
\end{align}