\chapter{绪论}
\section{常微分方程的基本概念}
\subsection{微分方程}
\begin{definition}[微分方程]\label{def: 1.2 diffeq}
    含有未知函数和未知函数的导数的方程
\end{definition}
\subsubsection{分类}
\noindent 按照定义域区分:
\begin{enumerate}
    \item 实值微分方程
    \item 复值微分方差
\end{enumerate}
\noindent 按照自变量个数区分:
\begin{enumerate}
    \item \textbf{常微分方程}
    单个自变量,未知函数是一元函数
    \item \textbf{偏微分方程}
    多个自变量,位置函数是多元函数
\end{enumerate}

\begin{definition}[微分方程的阶数]
    微分方程中出现的未知函数\textbf{最高阶}导数的阶数
\end{definition}

n阶微分方程:
    \begin{itemize}
        \item \textbf{隐式结构}
        \begin{equation}
            F(x,y,y',\dots,y^{(n-1)},y^{(n)})=0
        \end{equation}\label{eq:1.2 diffeq_v}
        \item  \textbf{显式结构}
        \begin{equation}
            y^{(n)}=f(x,y,\dots ,y^{(n-1)})
        \end{equation}\label{eq:1.2 diffeq_i}
    \end{itemize}
    
\subsubsection{线性与非线性方程}
\begin{definition}[n阶线性微分方程]\label{def:1.2 n-lindifeq}
    \begin{equation}
        y^{(n)}+a_1(x)y^{(n-1)}+\dots+a_{n-1}(x)y'+a_n(x)y=f(x)
    \end{equation}
    \textbf{注:}当$f(x)\equiv0$时,为n阶齐次线性微分方程,反之为n阶非齐次线性微分方程
\end{definition}

\subsubsection{微分方程的解}
\begin{definition}[解]\label{def:1.2 sol}
    使$F(x,y,y',\dots,y^{(n)})$恒成立的函数$y=\varphi(x)$
    \begin{equation*}
        F(x,\varphi(x),\dots,\varphi^{(n)}(x))\equiv0
    \end{equation*}
\end{definition}

\noindent \textbf{解的分类:}
\begin{itemize}
    \item \textbf{通解}\\
    n阶微分方程的解中含有n个任意常数,且这n个任意常数\textbf{独立}.
    \begin{equation*}
        y=\varphi(x,C_1,\dots C_n)
    \end{equation*}
    \item \textbf{特解}\\
    \textbf{不含}任意特定常数的解
\end{itemize}

\begin{itemize}
    \item \textbf{显式解}
    可以显式地写出$y=y(x)$
    \item \textbf{隐式解}
    $x,y$的关系只能通过关系式$\Phi(x,y)$表示
    \begin{example}
        一阶微分方程:
        \begin{equation}
            \frac{dy}{dx}=-\frac{x}{y}
        \end{equation}\label{eq:1.2 diffeq}
        有解:
        \begin{align}
            y&=\sqrt{1-x^2}\\
            y&=-\sqrt{1-x^2}
        \end{align}\label{eq:1.2 sol_v}
        也可用关系式:
        \begin{equation}
            x^2+y^2=1
        \end{equation}\label{eq:1.2 sol_i}
        表达,其中\ref{eq:1.2 sol_v}就是方程\ref{def: 1.2 diffeq}的显式解,\ref{eq:1.2 sol_i}是隐式解
    \end{example}
\end{itemize}

\begin{definition}[定解条件]
    为了求出微分方程一个特解所给出的条件
\end{definition}

\begin{definition}[定解问题]
    求微分方程满足定解条件的问题
\end{definition}

\begin{definition}[初值条件]
    对微分方程\ref{eq:1.2 diffeq_i},初值条件是指如下的n个条件:
    \begin{equation}
        \text{当}x=x_0\text{时},\quad y=y_0,\quad y'=y'_0,\cdots,y^{(n-1)}=y_0^{(n-1)}
    \end{equation}
\end{definition}

\begin{definition}[初值问题]
    当定值条件为初值条件时的定解问题
\end{definition}

\subsubsection{积分曲线与向量场}
\begin{definition}[积分曲线]\label{def:}
    一阶微分方程:
    \begin{equation}
        \frac{dy}{dx}=f(x,y)
    \end{equation}
    的解$y=\varphi(x)$表示$Oxy$平面上的一条曲线,称为微分方程的\textbf{积分曲线},而通解$y=\varphi(x,c)$表示平面上的一族曲线,称为微分方程的\textbf{积分曲线族},特解$\varphi(x_0)=y_0$则为过点$(x_0,y_0)$的一条积分曲线.
\end{definition}
\begin{note}
    积分曲线上过每一点的切线斜率$\frac{dy}{dx}$为方程右端$f(x,y)$在该处的值.反之,如果有一条曲线,其上每一点的切线斜率都等于$f(x,y)$,则此曲线为积分曲线.
\end{note}
\begin{figure}
    \centering
    \includegraphics[width=0.5\linewidth]{figure/Chapter1/Figure_1.png}
    \caption{方程}
    \label{fig:placeholder}
\end{figure}
\begin{definition}[向量场]
    
\end{definition}
\begin{definition}[线素场]
    
\end{definition}

\subsubsection{微分方程组}
\subsubsection{驻定与非驻定,动力系统}
\subsubsection{相空间和轨线}
\subsubsection{雅可比矩阵与函数相关性}
以方程$y''-y=0$为例通过\textbf{雅可比矩阵}分析常数$C_1,C_2$的独立性:
\begin{definition}[雅可比行列式]
    
\end{definition}